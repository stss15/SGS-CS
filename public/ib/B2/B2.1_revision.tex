\documentclass[a4paper, 12pt]{article}
\usepackage[landscape, top=1cm, bottom=1cm, left=1cm, right=1cm]{geometry}
\usepackage{fontspec}
\usepackage{xcolor}
\usepackage{tcolorbox}
\usepackage{listings}
\usepackage{enumitem}
\usepackage{array}
\usepackage{tabularx}
\usepackage{colortbl}
\usepackage{graphicx}

% --- Language Setup ---
\usepackage[english]{babel}

% --- Font Setup ---
\setmainfont{Noto Sans}
\setmonofont{Noto Sans Mono}

% --- Colours ---
\definecolor{codebg}{RGB}{248,248,248}
\definecolor{headerblue}{RGB}{0, 102, 204}
\definecolor{headergreen}{RGB}{0, 153, 76}
\definecolor{headerpurple}{RGB}{102, 0, 153}
\definecolor{headerteal}{RGB}{0, 128, 128}
\definecolor{headerred}{RGB}{204, 0, 0}
\definecolor{tipbg}{RGB}{255, 250, 230}
\definecolor{tipborder}{RGB}{255, 179, 0}
\definecolor{tableheader}{RGB}{230, 240, 255}

% --- Listing Style ---
\lstset{
    language=Python,
    basicstyle=\ttfamily\large,
    backgroundcolor=\color{codebg},
    keywordstyle=\color{headerblue}\bfseries,
    stringstyle=\color{headerred},
    commentstyle=\color{headergreen},
    numberstyle=\tiny\color{gray},
    showstringspaces=false,
    frame=none,
    aboveskip=10pt,
    belowskip=10pt,
    breaklines=true,
    morekeywords={True, False, None, as}
}

% --- TColorBox Setup ---
\tcbuselibrary{skins, breakable}

% Description Box (Top)
\newtcolorbox{descbox}[2][]{
    colback=white,
    colframe=#2,
    fonttitle=\bfseries\Large,
    title=#1,
    boxrule=1.5pt,
    rounded corners,
    arc=4mm,
    left=10pt, right=10pt, top=10pt, bottom=10pt,
    before skip=5pt,
    after skip=10pt
}

% Main Content Box (Middle)
\newtcolorbox{contentbox}[1][]{
    colback=white,
    colframe=gray!50,
    boxrule=1pt,
    rounded corners,
    arc=2mm,
    left=10pt, right=10pt, top=10pt, bottom=10pt,
    before skip=5pt,
    after skip=10pt,
    #1
}

% Tips Box (Bottom)
\newtcolorbox{tipbox}{
    colback=tipbg,
    colframe=tipborder,
    fonttitle=\bfseries\Large,
    title=Revision \& Exam Tips,
    coltitle=black,
    boxrule=1.5pt,
    rounded corners,
    arc=4mm,
    left=10pt, right=10pt, top=10pt, bottom=10pt,
    before skip=10pt,
    after skip=5pt
}

% --- Table Setup ---
\newcolumntype{L}{>{\raggedright\arraybackslash}X}
\renewcommand{\arraystretch}{1.4}

\begin{document}

% --- PAGE 1: DATA TYPES ---
\begin{descbox}[Data Types \& Variables]{headerblue}
    \textbf{Description:} Variables are containers for storing data values. In Python, variables are dynamically typed (you don't declare the type explicitly).
\end{descbox}

\begin{contentbox}
    \textbf{\Large Core Data Types}
    \begin{itemize}[itemsep=5pt]
        \item \textbf{\texttt{int}}: Whole numbers (e.g., \texttt{10}, \texttt{-5}).
        \item \textbf{\texttt{float}}: Decimal numbers (e.g., \texttt{3.14}, \texttt{2.0}).
        \item \textbf{\texttt{str}}: Text strings (e.g., \texttt{"Hello"}). \textit{Note: Python does NOT have a separate 'char' type.}
        \item \textbf{\texttt{bool}}: Logic values (\texttt{True}, \texttt{False}).
    \end{itemize}

    \vspace{0.5cm}
    \textbf{\Large Variable Swapping (The "Cup" Analogy)}
    \begin{lstlisting}
a = 5      # Cup A (Water)
b = 10     # Cup B (Tea)

temp = a   # 1. Pour 'a' into empty temp cup
a = b      # 2. Pour 'b' into 'a'
b = temp   # 3. Pour 'temp' into 'b'
    \end{lstlisting}
\end{contentbox}

\vfill

\begin{tipbox}
    \begin{itemize}[label=\textcolor{tipborder}{\textbullet}, itemsep=5pt]
        \item \textbf{Misconception:} Don't confuse assignment (\texttt{=}) with equality comparison (\texttt{==}).
        \item \textbf{Exam Tip:} If asked to "declare" a variable in pseudocode, Python just assigns a value (e.g., \texttt{score = 0}).
        \item \textbf{Watch Out:} \texttt{"5"} is a String, \texttt{5} is an Integer. You cannot add them directly!
    \end{itemize}
\end{tipbox}

\newpage

% --- PAGE 2: OPERATORS ---
\begin{descbox}[Operators]{headergreen}
    \textbf{Description:} Operators perform operations on variables and values. Python has specific symbols for arithmetic, comparison, and logic.
\end{descbox}

\begin{contentbox}
    \begin{tabularx}{\linewidth}{|c|L|L|}
        \hline
        \rowcolor{tableheader} \textbf{Type} & \textbf{Operator \& Description} & \textbf{Example} \\ \hline
        \textbf{Arithmetic} & 
        \texttt{+} Add, \texttt{-} Subtract, \texttt{*} Multiply & \texttt{5 + 2} $\rightarrow$ \texttt{7} \\ 
        & \texttt{/} Float Division (Decimal result) & \texttt{7 / 2} $\rightarrow$ \texttt{3.5} \\
        & \texttt{//} Floor Division (Integer result) & \texttt{7 // 2} $\rightarrow$ \texttt{3} \\
        & \texttt{\%} Modulus (Remainder) & \texttt{7 \% 2} $\rightarrow$ \texttt{1} \\
        & \texttt{**} Exponentiation (Power) & \texttt{2 ** 3} $\rightarrow$ \texttt{8} \\ \hline
        
        \textbf{Relational} & 
        \texttt{==} Equal to & \texttt{5 == 5} $\rightarrow$ \texttt{True} \\
        & \texttt{!=} Not equal to & \texttt{5 != 3} $\rightarrow$ \texttt{True} \\
        & \texttt{<} Less, \texttt{>} Greater & \texttt{2 < 5} $\rightarrow$ \texttt{True} \\
        & \texttt{<=} Less/Equal, \texttt{>=} Greater/Equal & \texttt{5 >= 5} $\rightarrow$ \texttt{True} \\ \hline
        
        \textbf{Logical} & 
        \texttt{and} (Both true) & \texttt{True and False} $\rightarrow$ \texttt{False} \\
        & \texttt{or} (At least one true) & \texttt{True or False} $\rightarrow$ \texttt{True} \\
        & \texttt{not} (Invert value) & \texttt{not True} $\rightarrow$ \texttt{False} \\ \hline
    \end{tabularx}
\end{contentbox}

\vfill

\begin{tipbox}
    \begin{itemize}[label=\textcolor{tipborder}{\textbullet}, itemsep=5pt]
        \item \textbf{Critical Difference:} \texttt{/} \textit{always} creates a float. \texttt{//} \textit{always} rounds down to an int.
        \item \textbf{Misconception:} \texttt{and} checks both sides; \texttt{or} stops early if the first one is True (short-circuiting).
        \item \textbf{Exam Tip:} Remember \textbf{BODMAS/BIDMAS} applies to programming too!
    \end{itemize}
\end{tipbox}

\newpage

% --- PAGE 3: STRINGS ---
\begin{descbox}[String Manipulation]{headerpurple}
    \textbf{Description:} Strings are sequences of characters. They are \textbf{immutable}, meaning you cannot change a character in place.
\end{descbox}

\begin{contentbox}
    \textbf{\Large Slicing Syntax: \texttt{string[start : stop : step]}}
    \begin{lstlisting}
txt = "Computer"
# Indices: 01234567

print(txt[0:3])   # "Com" (Indices 0, 1, 2 - STOP excluded)
print(txt[3:])    # "puter" (From 3 to end)
print(txt[-1])    # "r" (Last character)
    \end{lstlisting}

    \vspace{0.5cm}
    \textbf{\Large Common Methods}
    \begin{tabularx}{\linewidth}{|l|L|}
        \hline
        \textbf{Method} & \textbf{Function} \\ \hline
        \texttt{len(s)} & Returns number of characters (including spaces). \\ \hline
        \texttt{s.strip()} & Removes whitespace from start and end. \\ \hline
        \texttt{s.replace(old, new)} & Replaces all occurrences of a substring. \\ \hline
        \texttt{s + s} & Concatenation (joining strings). \\ \hline
    \end{tabularx}
\end{contentbox}

\vfill

\begin{tipbox}
    \begin{itemize}[label=\textcolor{tipborder}{\textbullet}, itemsep=5pt]
        \item \textbf{Common Error:} \texttt{txt[3]} gives the 4th character because we count from 0.
        \item \textbf{Exam Tip:} Slicing \texttt{[a:b]} includes \texttt{a} but \textbf{excludes} \texttt{b}.
        \item \textbf{Memory:} Strings don't change. \texttt{txt.upper()} returns a \textit{new} string; it doesn't change \texttt{txt} unless you reassign it.
    \end{itemize}
\end{tipbox}

\newpage

% --- PAGE 4: ERROR HANDLING ---
\begin{descbox}[Error Handling]{headerteal}
    \textbf{Description:} Exception handling prevents programs from crashing when runtime errors occur (like dividing by zero).
\end{descbox}

\begin{contentbox}
    \begin{lstlisting}
try:
    num = int(input("Enter number: "))
    result = 100 / num
except ValueError:
    # Runs if user types text instead of a number
    print("That is not a valid number!")
except ZeroDivisionError:
    # Runs if user types 0
    print("Cannot divide by zero!")
except Exception as e:
    # Generic catch-all (use sparingly)
    print(f"Unknown error: {e}")
finally:
    # ALWAYS runs (for cleanup)
    print("Execution finished.")
    \end{lstlisting}
\end{contentbox}

\vfill

\begin{tipbox}
    \begin{itemize}[label=\textcolor{tipborder}{\textbullet}, itemsep=5pt]
        \item \textbf{Best Practice:} Always catch specific errors (\texttt{ValueError}) before generic ones (\texttt{Exception}).
        \item \textbf{Exam Tip:} \texttt{finally} is useful for closing files or database connections.
        \item \textbf{Difference:} \textbf{Syntax Errors} happen before code runs. \textbf{Exceptions} happen \textit{during} execution.
    \end{itemize}
\end{tipbox}

\newpage

% --- PAGE 5: DEBUGGING ---
\begin{descbox}[Debugging \& Trace Tables]{headerred}
    \textbf{Description:} Debugging is finding/fixing errors. Trace tables are manual checks of logic; Breakpoints use the IDE to pause execution.
\end{descbox}

\begin{contentbox}
    \textbf{\Large Algorithm: Linear Search}
    \begin{lstlisting}
nums = [10, 50, 30]
target = 50
found = False
for i in range(len(nums)):
    if nums[i] == target:
        found = True
        break
    \end{lstlisting}

    \vspace{0.3cm}
    \textbf{\Large Trace Table (Manual Logic Check)}
    \begin{tabularx}{\linewidth}{|c|c|c|c|c|}
        \hline
        \rowcolor{tableheader} \textbf{step} & \textbf{i} & \textbf{nums[i]} & \textbf{nums[i] == target} & \textbf{found} \\ \hline
        Init & - & - & - & False \\ \hline
        Loop 1 & 0 & 10 & False & False \\ \hline
        Loop 2 & 1 & 50 & \textbf{True} & \textbf{True} \\ \hline
        Break & - & - & - & True \\ \hline
    \end{tabularx}
\end{contentbox}

\vfill

\begin{tipbox}
    \begin{itemize}[label=\textcolor{tipborder}{\textbullet}, itemsep=5pt]
        \item \textbf{Trace Tables:} Use these to test your \textit{logic} on paper before writing code.
        \item \textbf{Breakpoints:} Use these to pause code and inspect variable values at a specific moment.
        \item \textbf{Step Over vs Step Into:} "Over" runs the line; "Into" goes inside a function call.
    \end{itemize}
\end{tipbox}

\end{document}